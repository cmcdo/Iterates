\documentclass{article}
\usepackage{amsmath}
\usepackage{amssymb}
\usepackage{amsthm}
\usepackage[a4paper, total={6in, 8in}]{geometry}
\setlength\parindent{0pt}

% ShortCuts
\newcommand{\N}{\mathbb{N}}
\newcommand{\R}{\mathbb{R}}
\newcommand{\C}{\mathbb{C}}
\newcommand{\D}{\mathbb{D}}
\newcommand{\Hp}{\mathbb{H}}
\newcommand{\eps}{\epsilon}
\newcommand{\B}{\mathcal{B}}
\newcommand{\F}{\mathcal{F}}

%Environ
\newtheorem{theorem}{Theorem}
\newtheorem{corollary}[theorem]{Corollary}
\newtheorem{lemma}[theorem]{Lemma}
\newtheorem{definition}[theorem]{Definition}

\title{Concise Description of the Sequences of Conformal Iterates of the Unit Disk}

\author{Clayton McDonald}

\begin{document}
\maketitle
\section{Prerequisite Tools}

\begin{theorem}[Iterates of $\D$]
    Let $f:\D\to\D$ be a biholomorphism, that is bijective and holomorphic (hence conformal) self map of $\D$, then $f$ takes the form
    \begin{equation*}
        f(z) = e^{i\theta} \frac{z-a}{1-\overline{a}z}
    \end{equation*}
Where $a\in\D$ and $\theta\in\R$
\end{theorem}
We wish to fully explain the condition for convergence of the sequence of iterates of functions of this form.
The reason for doing so, is due to the Riemann mapping theorem if we can solve this problem on this domain, we will be able to fully understand
the iterates for any domain of $\C$. 
\begin{theorem}[Reimann Mapping Theorem]
Let $\Omega\subset\C$ be a domain of $\C$, then for all $z_0\in\Omega$ there exists some biholomorphism $F:\Omega\to\D$ such that $F(z_0) = 0$ and $F'(z_0)>0$
\end{theorem}
\begin{corollary}
    Any two domains of $\C$, say $\Omega_1$ $\Omega_2$ are conformally equivalent, meaning there exists some biholomorphism between them.
\end{corollary}
To see why, let $F_i$ be the biholomorphisms from $\Omega_i$ to $\D$, then clearly $F_2^{-1} \circ F_1:\Omega_1\to\Omega_2$ and clearly a biholomorphism \\ 
\begin{theorem}[The Hyperbolic Plane]
    Consider the metric space $\mathcal{D} = (\D, d_\D)$ where $d_\D$ is defined by
    \begin{equation*}
        d_\D (z, w) = 2 \tanh^{-1}\bigg|\frac{z-w}{1-z\overline{w}} \bigg|
    \end{equation*}
We call this this Hyperbolic Plane.
\end{theorem}
The key to this space is that the biholomorphisms of $\D$ are the isometries of $\mathcal{D}$, this fact we will exploit to study the sequences of iterates of the biholomorphisms of $\D$
\section{Iterates on $\D$}
\end{document}
