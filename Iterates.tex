\documentclass{article}
\usepackage{amsmath}
\usepackage{amssymb}
\usepackage{amsthm}
\usepackage[a4paper, total={6in, 8in}]{geometry}
\setlength\parindent{0pt}

% ShortCuts
\newcommand{\N}{\mathbb{N}}
\newcommand{\R}{\mathbb{R}}
\newcommand{\C}{\mathbb{C}}
\newcommand{\D}{\mathbb{D}}
\newcommand{\Hp}{\mathbb{H}}
\newcommand{\eps}{\epsilon}
\newcommand{\B}{\mathcal{B}}
\newcommand{\F}{\mathcal{F}}

%Environ
\newtheorem{theorem}{Theorem}
\newtheorem{corollary}[theorem]{Corollary}
\newtheorem{lemma}[theorem]{Lemma}
\newtheorem{definition}[theorem]{Definition}

\title{Concise Description of the Sequences of Conformal Iterates of the Unit Disk}

\author{Clayton McDonald}

\begin{document}
\maketitle
\section{Prerequisite Tools}

\begin{theorem}[Iterates of $\D$]
    Let $f:\D\to\D$ be a biholomorphism, that is bijective and holomorphic (hence conformal) self map of $\D$, then $f$ takes the form
    \begin{equation*}
        f(z) = e^{i\theta} \frac{z-a}{1-\overline{a}z}
    \end{equation*}
Where $a\in\D$ and $\theta\in\R$
\end{theorem}
We wish to fully explain the condition for convergence of the sequence of iterates of functions of this form.
The reason for doing so, is due to the Riemann mapping theorem if we can solve this problem on this domain, we will be able to fully understand
the iterates for any domain of $\C$. 
\begin{theorem}[Reimann Mapping Theorem]
Let $\Omega\subset\C$ be a domain of $\C$, then for all $z_0\in\Omega$ there exists some biholomorphism $F:\Omega\to\D$ such that $F(z_0) = 0$ and $F'(z_0)>0$
\end{theorem}
\begin{corollary}
    Any two domains of $\C$, say $\Omega_1$ $\Omega_2$ are conformally equivalent, meaning there exists some biholomorphism between them.
\end{corollary}
To see why, let $F_i$ be the biholomorphisms from $\Omega_i$ to $\D$, then clearly $F_2^{-1} \circ F_1:\Omega_1\to\Omega_2$ and clearly a biholomorphism \\ 
\begin{theorem}[The Hyperbolic Plane]
    Consider the metric space $\mathcal{D} = (\D, d_\D)$ where $d_\D$ is defined by
    \begin{equation*}
        d_\D (z, w) = 2 \tanh^{-1}\bigg|\frac{z-w}{1-z\overline{w}} \bigg|
    \end{equation*}
We call this this Hyperbolic Plane.
\end{theorem}
The key to this space is that the biholomorphisms of $\D$ are the isometries of $\mathcal{D}$, this fact we will exploit to study the sequences of iterates of the biholomorphisms of $\D$
\section{Iterates on $\D$}
First, setting $f(z) = z$ and solving we will find a quadratic equation, hence these biholomorphisms have at most two unique fixed points.
\begin{theorem}
    Let $f$ be a biholomorphism of $\D$, if we have a fixed point $z^*\in\D$ then the sequence defined by $z_n = f(z_{n-1})$ with $z_0\in\D\setminus\{z^*\}$ will not converge.
\end{theorem}
\begin{proof}
    We noted before that $f$ must be a hyperbolic isometry, we know that these sequences of iterates must only converge to fixed points, however
    \begin{equation*}
        d_\D(f(z_n), f(z^*)) = d_\D(z_n, z^*) = d_\D(z_0, z^*) = K
    \end{equation*}
    From this one may see, $d_\D(z_n, z^*) = 2 \tanh^{-1}\bigg|\frac{z_n - z^*}{1-\overline{z^*}z_n} \bigg|  = K$ which we then see 

    \begin{equation*}
        d_E(z_n, z^*) = K'|1-\overline{z^*}z_n|
    \end{equation*}
    We cannot have convergence as the RHS of this equation does not tend to zero. For $w, z\in\D$ clearly $|w||z| < 1$ so the term $|1-\overline{z^*}z_n|$ is bounded below by some constant $c\in\R^+$
\end{proof}
\begin{theorem}
    Let $f:\D\to\D$ be a biholomorphism of $\D$ in the usual form, then the sequence of iterates converges if and only if $|a| > \sin(\theta/2)$
\end{theorem}
\begin{proof}

\end{proof}
\end{document}
